\documentclass[12pt]{book}

\usepackage[margin=1.0in]{geometry}
\usepackage{amsmath,amsthm,amssymb}

\newcommand{\rr}{\mathbb{R}}
\newcommand{\cc}{\mathbb{C}}
\newcommand{\nn}{\mathbb{N}}
\newcommand{\zz}{\mathbb{Z}}

\newtheorem{ex}{Exercise}

\title{Solutions to the Algebra Preliminary Exam Study Guide}
\author{James (Jimmy) Thorne}
\begin{document}
	\maketitle
	
	\begin{ex}
		An element is called idempotent if $x^2=x$.  Show that if each element $a$
		in the ring $R$
		is idempotent, then $R$ is a commutative ring.  (Note: $R$ is called a
		boolean ring)
	\end{ex}

	\begin{proof}
		Let $x,y\in R$.
		
		\begin{align}
			(x+y)^2 &= (x+y)(x+y) \\
			&= x^2 + xy + yx + y^2 \\
			&= x + xy + yx +y 
		\end{align}
		
		We also have the $(x+y)^2=x+y \, (4)$ by the idempotent property,
		so we combine (3) and (4)  to get
		
		$$x + y = x + xy + yx + y \implies 0 = xy + yx \implies -xy = yx $$
		
		By squaring both sides of this last expression we get
		
		$$xy = yx$$
		
		
	\end{proof}
	
\end{document}