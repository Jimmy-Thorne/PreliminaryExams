\documentclass[12pt]{book}

\usepackage[margin=1.0in]{geometry}
\usepackage{amsmath,amsthm,amssymb,titling}

\newcommand{\rr}{\mathbb{R}}
\newcommand{\cc}{\mathbb{C}}
\newcommand{\nn}{\mathbb{N}}
\newcommand{\zz}{\mathbb{Z}}

\newcommand{\subtitle}[1]{%
	\posttitle{%
		\par\end{center}
	\begin{center}\large#1\end{center}
	\vskip0.5em}%
}

\title{Algebra Preliminary Exam Study Guide}
\subtitle{NDSU}
\author{James (Jimmy) Thorne}
\date{\today}

\theoremstyle{definition}
\newtheorem{define}{Definition}
\newtheorem{ex}{Exercise}


\begin{document}
	
	\maketitle
	
	\chapter{Rings}
	
	\begin{define}
		A \underline{Ring} R is a set together with two binary operations $+$ and $\times$
		such that
		\begin{enumerate}
			\item $(R,+)$ is an abelian group,
			\item $\times$ is associative
			\item $(a+b)\times c = (a\times c) + (b\times c)$ for all $a,b,c\in R$.	
		\end{enumerate}
	\end{define}

	\begin{define}
		A ring is \underline{commutative ring} if $\times$ is commutative and is said to 
		\underline{have identity} if there is a $1\in R$ such that 
		
		$$ 1\times a = a \times 1 =a \hspace{0.5in} \mbox{for all } a\in R$$
	\end{define}
	
	\begin{ex}
		An element is called idempotent if $x^2=x$.  Show that if each element $a$
		in the ring $R$
		is idempotent, then $R$ is a commutative ring.  (Note: $R$ is called a
		boolean ring)
	\end{ex}

	We can add more structure to the rings with the following definition.
	
	\begin{define}
		Let $R$ be a ring with identity 1, where $\neq 0$.  We say $R$ is a 
		\underline{division ring} if every nonzero element $a\in R$ has a
		multiplicative inverse.  If $R$ is commutative, then we say $R$ is a 
		\underline{field}.
	\end{define}

	
\end{document}