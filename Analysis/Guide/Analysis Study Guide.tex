\documentclass[12pt]{book}

\usepackage[margin=1.0in]{geometry}
\usepackage{amsmath,amsthm,amssymb,titling}

\newcommand{\rr}{\mathbb{R}}
\newcommand{\cc}{\mathbb{C}}
\newcommand{\nn}{\mathbb{N}}
\newcommand{\zz}{\mathbb{Z}}

\newcommand{\subtitle}[1]{%
	\posttitle{%
		\par\end{center}
	\begin{center}\large#1\end{center}
	\vskip0.5em}%
}

\title{Analysis Preliminary Exam Study Guide}
\subtitle{NDSU}
\author{James (Jimmy) Thorne}
\date{\today}

\theoremstyle{definition}
\newtheorem{define}{Definition}
\newtheorem{ex}{Exercise}


\begin{document}
	\maketitle
	
	\chapter{Measure Theory}
	
	\section{$\sigma$-Algebras}
	
	\begin{define}
		Let $X$ be a set.  $\mathcal{A}\subset \mathcal{P}(X)$ is an \underline{algebra} if 
		\begin{enumerate}
			\item[i. ] For $A\in \mathcal{A}, A^c \in \mathcal{A}$
			\item[ii. ] If $A,B\in \mathcal{A}, A\cup B \in \mathcal{A}$
		\end{enumerate}
	We say that $\mathcal{A}$ is a \underline{$\sigma$-algebra} if
		\begin{enumerate}
			\item[iii. ] If $A_j$ is a countable collection of sets in $\mathcal{A}$, then $\bigcup_{j\in \nn} A_j \in \mathcal{A}$.
	\end{enumerate} 
\end{define}

	\begin{ex}[May 19]
		Let $\mathcal{A}$ be an algebra of sets that is closed under countable increasing unions.  Show that
		$\mathcal{A}$ is a $\sigma$-algebra.
	\end{ex}

	\begin{ex}[Jan 18]
		Let $\mathcal{S}$ be the collection of all subsets of $[0,1)$ which can be written as a finite union
		of intervals of the form $[a,b) \subseteq [0,1).$  Show that $\mathcal{S}$ is an algebra of sets, but is not 
		a $\sigma$-algebra.
	\end{ex}
	
	\begin{define}
	Since the intersection of $\sigma$-algebra is itself a $\sigma$-algebra, for
	a given collection of sets $\mathcal{E}$  we can define $\mathnormal{M}(\mathcal{E})$ as the intersections of
	all $\sigma$-algebras containing  $\mathcal{E}$.  We say $M(\mathcal{E})$ is the \underline{$\sigma$-algebra generated by 
	$\mathcal{E}$.}  If $\mathcal{E}$ is all the open sets on $X$, then $M(\mathcal{E})$ is the \underline{Borel $\sigma$-algebra.}
	\end{define}

	The Borel $\sigma$-algebra on $\rr$ can be generated with several different sets.  For example, one could use closed sets.
	Since the complement of every open set is closed (and vice versa) we generate the same $\sigma$-algebra.\\
	
	\noindent\textbf{Fact: } If $\mathcal{E} \subseteq M(\mathcal{F})$, then $M(\mathcal{E}) \subseteq M(\mathcal{F})$
	
	\begin{ex}
		Show that the closed rays $(-\infty,a]$ will generate the Borel $\sigma-$algebra. 
	\end{ex}
\end{document}